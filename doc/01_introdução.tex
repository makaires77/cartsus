\section{Introdução}
Inteligência Analítica é o campo que engloba o uso de técnicas avançadas de análise de dados, tais como mineração de dados, machine learning, processamento de linguagem natural, e estatísticas, para descobrir padrões, gerar insights, tomar decisões informadas e prever futuros comportamentos e tendências. Essa prática geralmente envolve o uso de tecnologias e software especializados para coletar, processar, limpar, analisar e visualizar dados de várias fontes e em grandes quantidades. O objetivo é ajudar indivíduos, empresas e organizações a tomar decisões baseadas em dados, em vez de apenas na intuição ou na experiência passada, particularmente quando a commplexidade do assunto demanda a construção coletiva de uma decisão que não é acessível a um agente isolado. A inteligência analítica pode ser aplicada em diversas áreas como saúde, finanças, operações, recursos humanos, marketing e muito mais. Na área de finanças, por exemplo, pode ser usada para detectar fraudes, gerenciar riscos, otimizar portfólios de investimentos. Já em marketing, pode-se tratar sobre tarefas como segmentar os clientes, prever a probabilidade de churn, otimizar campanhas publicitárias, entre muitas outras aplicações práticas.

A programação dinâmica é um poderoso método utilizado para resolver problemas de otimização que possuem uma estrutura de sobreposição de subproblemas. Em termos simples, a programação dinâmica divide um problema maior em subproblemas menores, resolve cada um deles apenas uma vez e armazena seus resultados em uma tabela (memória), de onde é possível reconstruir a solução para o problema original. Em grafos, a programação dinâmica tem uma ampla aplicação, especialmente quando se trata de problemas de caminho mais curto, contagem de caminhos, ordenação topológica e muito mais. No contexto de sistemas de recomendação, a aplicação da programação dinâmica pode ter vários ângulos, dependendo do tipo de problema que se está tentando resolver. Dentro do contexto de sistemas de recomendação baseados em conteúdo, a programação dinâmica pode ser usada para comparar sequências de opções ou escolha de itens. Isso poderia ser útil, por exemplo, para comparar diferentes lista de opções, como em opções de alternativas de tratamentos, escolha de medicamentos, delineamento de rotas tecnológicas, ou até questões mais corriqueiras de nossa vida cotidiana como organizar listas de reprodução de músicas ou sequências de filmes/séries de acordo com a preferência ou necessidade específica do usuário.

Para aplicações relacionadas com sistemas de recomendação com uso de estruturas de grafos (por exemplo, sistemas que usam dados de redes sociais, como citações, coautoria, colaborações de qualquer forma), a abordagem com grafos pode ser usada para identificar comunidades, detectar hubs de influência e analisar a estrutura geral da rede. A programação dinâmica também pode ser usada em combinação com técnicas de aprendizado de máquina em grafos, do inglês \textit{Graph Machine Learning} (GML), como os modelos de Markov de ordem superior. Nesses casos, a programação dinâmica pode ser usada para resolver o problema de otimização subjacente ao aprendizado do modelo, que é frequentemente um problema de maximização de verossimilhança.

A aprendizagem por reforço (RL) é um ramo da inteligência artificial que oferece um arcabouço para a aprendizagem autônoma, onde um agente aprende a tomar decisões otimizadas pela interação com seu ambiente~\cite{SuttonBarto1998}. A incorporação de RL em GML traz um novo espectro de oportunidades e desafios. A representação em grafo oferece uma maneira natural de modelar a estrutura relacional complexa e os padrões de interação presentes em muitos problemas do mundo real, tornando-a especialmente relevante no campo da saúde e das ciências da vida~\cite{Zhang2020}. 

No domínio das ciências da vida e ciências da saúde, os nós do grafo podem representar entidades como proteínas, genes, pacientes, etc., e as arestas podem representar relações entre essas entidades (por exemplo, interações proteína-proteína, similaridade genética, etc.). Aprendizado de máquina em grafos permite a detecção de padrões complexos dentro desses dados relacionais. A aplicação de RL em GML tem mostrado promessa em várias aplicações, como a descoberta de medicamentos. Por exemplo, um agente de RL poderia aprender a projetar novas moléculas com propriedades desejadas navegando no espaço de todas as possíveis moléculas como um grafo, onde as ações representam modificações moleculares~\cite{You2018}. Outra aplicação possível é a aprendizagem de protocolos de tratamento ótimos para pacientes com base em seu histórico médico e estado atual, modelado como um grafo de paciente-tempo. As ações poderiam representar diferentes intervenções de tratamento, e a recompensa seria baseada no prognóstico do paciente~\cite{Weng2020}.